% !TeX root = ../main.tex

\section{进阶使用}
% \begin{frame}[fragile]{参考文献与扩展阅读}
% \newcommand{\BOOK}[1]{\textbf{#1}}
% \newcommand{\TAG}[1]{\textsc{[#1]}}
% \newcommand{\URL}[1]{\scalebox{0.92}[1]{\url{#1}}}
% \begin{multicols}{2}
%   \begin{thebibliography}{99}
%     \bibitem{}
%       \textsc{Knuth D E}.
%       \BOOK{The \TeX book: Computers \& Typesetting, volume C} \TAG{M}, 1984.
%       \newblock Addison--Wesley Publishing Company, Boston
%     \bibitem{}
%       \textsc{Bringhurst R}.
%       \BOOK{The Elements of Typographic Style, version 4.3} \TAG{M}, 2019.
%       \newblock Hartley \& Marks Publishers, Vancouver
%     \bibitem{}
%       刘海洋.
%       \BOOK{\LaTeX{} 入门} \TAG{M}, 2013.
%       \newblock 北京:电子工业出版社
%     \bibitem{}
%       \jatext{高冈昌生}.
%       刘庆~译,陈嵘~监修.
%       \BOOK{西文排版:排版的基础和规范} \TAG{M}, 2016.
%       \newblock 北京:中信出版集团
%     \bibitem{}
%       \textsc{Oetiker T}, \textsc{Partl H}, \textsc{Hyna I} and \textsc{Schlegl E}.
%       \CTeX{} 开发小组~译.
%       \BOOK{一份(不太)简短的 \LaTeXe{} 介绍:或 111 分钟了解 \LaTeXe{}} \TAG{EB/OL}, 2023.
%       \newblock \URL{https://ctan.org/pkg/lshort-zh-cn}
%     \bibitem{}
%       黄新刚(包太雷).
%       \BOOK{\LaTeX{} Notes: 雷太赫排版系统简介(第二版)} \TAG{EB/OL}, 2021.
%       \newblock \URL{https://github.com/huangxg/lnotes}
%     \bibitem{}
%       汪彧之,陈晟祺.
%       \BOOK{清华大学图书馆:如何使用 \LaTeX{} 排版论文} \TAG{EB/OL}, 2023.
%       \newblock \URL{https://github.com/tuna/thulib-latex-talk}
%     \bibitem{}
%       吴伟健,李子龙,钱宇超.
%       \BOOK{上海交通大学图书馆:如何使用 \LaTeX{} 排版论文\,/\,幻灯片} \TAG{EB/OL}, 2023.
%       \newblock \URL{https://github.com/sjtug/sjtulib-latex-talk}
%     \bibitem{}
%       刘海洋.
%       \BOOK{\LaTeX{} 快速入门} \TAG{EB/OL}, 2020.
%       \newblock Video: \href{https://www.bilibili.com/video/BV1s7411U7Pr}{\faVideo}
%       % Old version: https://bbs.pku.edu.cn/attach/e7/f2/e7f2bb698b9c3672/tex_intro_talk.pdf
%     \bibitem{}
%       林莲枝.
%       \BOOK{漫谈 \LaTeX{} 排版常见概念误区:别把 \LaTeX{} 当 Word 用!}\TAG{EB/OL}, 2018.
%       \newblock Video: \href{https://www.bilibili.com/video/BV1r4411o7KJ}{\faVideo}\quad
%         PDF: \href{http://static.latexstudio.net/wp-content/uploads/2018/03/LianTze-presentation-0320-forReading.pdf}{\faDownload}
%     \bibitem{}
%       Wikibooks.
%       \BOOK{\LaTeX{}---Wikibooks, The Free Textbook Project} \TAG{EB/OL}.
%       \newblock \URL{https://en.wikibooks.org/wiki/LaTeX}
%     \bibitem{}
%       Overleaf.
%       \BOOK{Overleaf Documentation} \TAG{EB/OL}.
%       \newblock \URL{https://www.overleaf.com/learn}
%     \bibitem{}
%       \LaTeX{} project.
%       \BOOK{Learn\LaTeX.org} \TAG{EB/OL}.
%       \newblock \URL{https://www.learnlatex.org}
%   \end{thebibliography}
% \end{multicols}
% \end{frame}

\begin{frame}{Git}
\begin{itemize}
  \item<+-> 版本管理的必要性
    \begin{itemize}
      \item 远离``初稿、第二稿、第三稿……终稿、终稿(打死也不改了)''
      \item 有底气做大范围修改、重构
      \item 方便与他人协同合作
    \end{itemize}

  \item<+-> 基本用法
    \begin{itemize}
      \item 把大象放进冰箱:|git init|、|git add|、|git commit|
      \item 时空穿梭:|git reset|、|git restore|、|git revert|
      \item 平行宇宙:|git branch|、|git switch|、|git rebase|
      \item 推荐用 VS Code 等进行可视化操作
      \item 参考链接:\link{https://git-scm.com/book}
        \link{https://ndpsoftware.com/git-cheatsheet.html}
        \link{https://nvie.com/posts/a-successful-git-branching-model/}
    \end{itemize}

  \item<+-> GitHub \link[\faGithub]{https://github.com}
    \begin{itemize}
      \item 远程 Git 仓库
      \item Clone \& fork
      \item Issues \& pull requests
      \item<+-> \alert{提醒:教育验证后可以有更多优惠}
    \end{itemize}
\end{itemize}
\end{frame}

\begin{frame}[standout]
  FAQ
\end{frame}

\begin{frame}{FAQ}
  \begin{itemize}
    \item \textbf{ \sout{网上抄的}神奇命令\cs{abc}怎么用不了?}\pause

    \item[] \begin{itemize}
      \item 在\TeX{}Studio 的 |cwl| 库搜索依赖宏包 \link[\faGithub]{https://github.com/texstudio-org/texstudio/tree/master/completion}
    \end{itemize}\pause

    \item \textbf{ 我不知道宏包 \pkg{xxx} 的用法?}\pause

    \item[] \begin{itemize}
      \item |texdoc -l xxx| 查看宏包文档
    \end{itemize}\pause

    \item \textbf{ 我想了解宏包 \pkg{xxx} 中命令\cs{abc}的实现?}\pause

    \item[] \begin{itemize}
      \item |latexdef -p xxx abc| 查看命令定义
    \end{itemize}\pause

    \item \textbf{ 我想知道宏包 \pkg{xxx} 的实现?}\pause

    \item[] \begin{itemize}
      \item |kpsewhich -all xxx.sty| 查看宏包路径以及优先级
    \end{itemize}\pause

    \item \textbf{ 我还想知道…… }\pause

    \item[] \begin{itemize}
      \item Google (in English) \link[\faGoogle]{https://www.google.com}
      \item \TeX{}--\LaTeX{} Stack Exchange \link{https://tex.stackexchange.com}
      \item C\TeX{} 临时论坛 \link[\faGithub]{https://github.com/CTeX-org/forum}
      \item 大语言模型 LLMs: ChatGPT, DeepSeek, Claude, Gemini,...
      \item 协会 QQ 群
    \end{itemize}
  \end{itemize}
\end{frame}


\begin{frame}{学有余力?}
\begin{itemize}
  \item 文档翻译
    \begin{itemize}
      \item \pkg{lshort-zh-cn} \link{https://github.com/CTeX-org/lshort-zh-cn}
      \item learnlatex.org/zh \link{https://github.com/CTeX-org/learnlatex.github.io}
      \item Wikibooks \link{https://zh.wikibooks.org/wiki/LaTeX}
    \end{itemize}

  \item 宏包开发与维护
    \begin{itemize}
      \item 不妨先从修 typo 开始
      \item 参与讨论,你的经验也可以解他人之忧
        \link{\thesisgithublink/discussions}
        \link{https://github.com/stone-zeng/fduthesis/discussions}
        \link{https://github.com/tuna/thuthesis/discussions}
        \link{https://github.com/sjtug/SJTUThesis/discussions}
      \item 欢迎参与维护 \pkg{hustthesis}, \pkg{hustvisual}
    \end{itemize}

  \item \sout{来当主讲人}
\end{itemize}
\end{frame}


\begin{frame}{关于}
  \begin{columns}[c]
    \begin{column}{.7\textwidth}
      \begin{itemize}
        \item 本幻灯片源码:
          \begin{itemize}
            \item \href{\githublink}{\faGithub\githublinkpath}
          \end{itemize}
        \item 本幻灯片参考\zhparen{了许多}:
          \begin{itemize}
            \item \href{https://github.com/stone-zeng/latex-talk}{\faGithub/stone-zeng/latex-talk}
            \item \href{https://github.com/tuna/thulib-latex-talk}{\faGithub/tuna/thulib-latex-talk}
          \end{itemize}
        \item 本幻灯片下载:
          \begin{itemize}
            \item GitHub Releases \link{\githublink/releases}
          \end{itemize}
        \item 许可证:CC BY-SA 4.0 \faCreativeCommons\,\faCreativeCommonsBy\,\faCreativeCommonsSa
      \end{itemize}
    \end{column}
    \begin{column}{.3\textwidth}
      \qrcode[hyperlink, height=\linewidth]{\githublink/releases}
    \end{column}
  \end{columns}
\end{frame}

\begin{frame}[standout]
  \begin{center}
    \Huge
    感谢
  \end{center}
\end{frame}
