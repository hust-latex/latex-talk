% !TeX root = ../main.tex

\section{\LaTeX{}入门}

\begin{frame}{\secname}
  \centering
  \smartdiagram[descriptive diagram]{
    {\textbf{工具链}, {源码解析+文档生成(如PDF)\\如\XeLaTeX,\BibTeX}},
    {\textbf{编辑器}, {代码编写界面\\高亮 + 补全 + 片段 + $\cdots$\\如VSCode,\TeX{}Studio,Vim}},
    {\textbf{发行版}, {工具链 + 宏包 + 字体 + 手册\\如\TeX{} Live, \MiKTeX{}}},
  }
\end{frame}

\begin{frame}{什么是工具链?}
  包含\TeX{}引擎(程序)和支持工具(程序),\\
  以文本文件为介质的程序间协同排版处理流水线
  \pause
  \centering
  % !TeX root = ../main.tex

% Copyright (c) 2025, HUANG Yuxi
% Released under the CC BY-SA 4.0 license.

\resizebox{\linewidth}{!}{
\begin{tikzpicture}[node distance=1.3cm and 2.8cm]
\tikzstyle{process} = [rectangle, rounded corners, minimum width=4.2cm, minimum height=1.1cm, align=center, draw=black, fill=blue!10]
\tikzstyle{files} = [chamfered rectangle, minimum width=4.2cm, minimum height=1cm,align=center, draw=black, fill=gray!20]
\tikzstyle{support} = [rectangle, rounded corners, minimum width=3.2cm, minimum height=1cm, align=center, draw=black, fill=green!20, dashed]
\tikzstyle{startstop} = [ellipse, minimum width=4cm, minimum height=1cm, align=center, draw=black, fill=red!20]
\tikzstyle{automation} = [diamond, aspect=2, minimum width=4cm, align=center, draw=black, fill=yellow!30, dashed]
\tikzstyle{arrow} = [thick, ->, >=Stealth]
\tikzstyle{optionalarrow} = [thick, ->, >=Stealth, dashed]
\tikzstyle{controlarrow} = [thick, ->, >=Stealth, dotted, double, double distance=3pt]
\tikzstyle{optionalprocess} = [process, dashed]

% Main flow
\node (source) [startstop] {.tex, .bib等源文件};

\node (engine) [process, below=of source] {\TeX/\LaTeX 引擎\\\pdfTeX / \XeLaTeX / ...};

\node (aux) [files, below=of engine] {中间文本文件\\
\scriptsize
\begin{tabular}{l}
.aux: 引用、标签 \\
.toc: 目录页码 \\
.bbl: 文献列表 \\
...
\end{tabular}
};

\node (rerun) [optionalprocess, below=of aux] {再次运行引擎\\
\scriptsize
以处理交叉引用、目录、索引};

\node (output) [startstop, below=of rerun] {最终输出\\PDF / DVI};

% Support tools
\node (index) [support, right=4cm of engine] {其他外部支持工具\\MakeIndex / xindy / ...};
\node (bib) [support, left=4cm of engine] {文献处理\\\BibTeX / \biber};

% Automation tools
\node (latexmk) [automation, left=4cm of source] {latexmk};
\node (arara) [automation, right=4cm of source] {arara};

\draw [arrow] (source) -- node[right] {手动运行} (engine);
\draw [arrow] (engine) -- node[right] {初次编译,生成中间文件} (aux);
\draw [optionalarrow] (aux) -- node[right] {处理引用/索引后再编译} (rerun);
\draw [optionalarrow, out=10, in=350, loop] (rerun) to node[right] {$\times 2\sim3$} (rerun);
\draw [arrow, bend right=60] (engine) to (output);
\draw [optionalarrow] (rerun) -- node[right] {生成输出文件} (output);

\draw [optionalarrow] (engine) -- node[above] {生成 .aux/.bib 数据} (bib);
\draw [optionalarrow] (engine) -- node[above] {.idx/.glo 等中间文件} (index);
\draw [optionalarrow] (bib) |- node[right, yshift=2ex] {.bbl 文献列表输出} (aux);
\draw [optionalarrow] (index) |- node[left, yshift=2ex] {.ind/.gls 等中间文件输出} (aux);

% Automation tools
\draw [controlarrow] (latexmk) -- node[above left, yshift=-3ex] {自动增量检测并多次调用引擎} (engine);
\draw [controlarrow] (arara) -- node[above right, yshift=-3ex] {基于注释规则控制编译流程} (engine);

\node[draw, rectangle, fill=white, anchor=south west, yshift=0.5cm, xshift=0.5cm] at (current bounding box.south west) {
  \begin{tabular}{ll}
    \tikz\node[startstop,scale=0.5] {}; & 起点/终点 \\
    \tikz\node[process,scale=0.5] {}; & 处理步骤 \\
    \tikz\node[files,scale=0.5] {}; & 中间文件 \\
    \tikz\node[optionalprocess,scale=0.5] {}; & 可选步骤 \\
    \tikz\node[support,scale=0.5] {}; & 可选支持工具 \\
    \tikz\node[automation,scale=0.5] {}; & 可选自动化工具 \\
  \end{tabular}
};


\node[draw, rectangle, fill=white, anchor=south east, yshift=0.5cm, xshift=-0.5cm] at (current bounding box.south east) {
  \begin{tabular}{ll}
    \tikz\draw[arrow] (0,0) -- (0.8,0); & 必要步骤 \\
    \tikz\draw[optionalarrow] (0,0) -- (0.8,0); & 可选步骤 \\
    \tikz\draw[controlarrow] (0,0) -- (0.8,0); & 自动控制 \\
  \end{tabular}
};

\end{tikzpicture}
}

\end{frame}

\begin{frame}{什么是\TeX{}引擎?}

  \TeX{}引擎是将\TeX{}源文件(文本+宏)编译成目标文件(如PDF)的程序。
  \pause{}例如:
  \begin{itemize}
    \item \TeX{}:最初的引擎,生成 |.dvi| 文件,需要外部程序 |dvipdf| 转换为 PDF
    \item \pdfTeX{}:直接生成 PDF,支持 micro-typography
    \item \XeTeX{}:支持 Unicode、OpenType 与复杂文字编排(CTL)
    \item \LuaTeX{}:支持 Unicode、OpenType,内联 Lua
    \item (u)p\TeX{}:日本方面推动,生成 |.dvi|,(支持 Unicode)
    \item Ap\TeX{}:底层 CJK 支持,内联 Ruby,Color Emoji \link[\faGithub]{https://github.com/clerkma/ptex-ng}
  \end{itemize}
\end{frame}


\begin{frame}{选择编辑器}
\begin{itemize}
  \item<+-> 专用型
    \begin{itemize}
      \item \TeX{}works:\TeXLive{} 自带 \faWindows{} \faApple{} \faLinux{}
      \item \TeX{}studio:功能丰富,对新手友好 \faWindows{} \faApple{} \faLinux{}
      \item \TeX{}Shop:\MacTeX{} 自带 \faApple{}
      \item WinEdt:功能丰富,收费 \faWindows{}
    \end{itemize}

  \item<+-> 通用型
    \begin{itemize}
      \item Visual Studio Code:配合 LaTeX Workshop 插件
      \item Sublime Text:需要收费
      \item Vim/Neovim:配合 Vim\TeX{} 插件
    \end{itemize}

  \item<+-> 编辑器对比:
    \link{https://tex.stackexchange.com/q/339}
    \link{https://en.wikipedia.org/wiki/Comparison_of_TeX_editors}
    \link{https://www.zhihu.com/question/19954023}
\end{itemize}
\end{frame}

\begin{frame}{\TeX{} 发行版}
  \begin{itemize}
    \item 发行版是\TeX{}工具链、宏包、字体、手册的集合 \pause
    \item 编译更快 \sout{(你得有更快的设备)}
    \item 自由的版本控制
    \item<+-> 自定义编译流程
    \item<+-> \textbf{\TeXLive{}}(\textbf{\MacTeX{}})
      \begin{itemize}
        \item 大而全,每年需手动更新
        \item 完整版占用空间较大(2025版约 \qty{10}{GB})
        \item 适合新手以及懒得折腾的用户
      \end{itemize}
    \item<+-> \textbf{\MiKTeX{}}
      \begin{itemize}
        \item 小而精,滚动更新
        \item 安装过程需要联网下载
        \item 部分行为与 \TeXLive{} 不同
      \end{itemize}
  \end{itemize}
\end{frame}

\begin{frame}{安装\TeX{} 发行版}
  \begin{itemize}
    \item 下载:华中科技大学开源镜像站 \link{https://mirrors.hust.edu.cn/CTAN/systems/texlive/Images/} \pause
    \item \faWindows{} Windows:
      \begin{itemize}
        \item 新手建议安装完整版 \TeXLive{}
        \item \sout{无脑}一直点下一步即可
        \item 保持耐心,普通性能电脑耗时最长可达 \qty{1}{h}
        \item 安装失败请先检查中文用户名或路径
      \end{itemize} \pause
    \item \faLinux{} Linux:
      \begin{itemize}
        \item 默认软件源通常落后版本,可以考虑 Vanilla \TeXLive{}
        \item GUI 安装界面需要 |perl-tk| 等
        \item 环境变量、|fontconfig|、dummy package 配置
      \end{itemize}
    \item \faApple{} macOS:
      \begin{itemize}
        \item 推荐 Homebrew\link[\faBeer]{https://brew.sh}: |brew install --cask mactex|
        \item 手动安装 \MacTeX{} 或者 \TeXLive{}
        \item 也可以使用 MacPorts
      \end{itemize}
    \item 手把手的教程:\link{https://www.ctan.org/pkg/install-latex-guide-zh-cn}
  \end{itemize}
\end{frame}


\begin{frame}{\TeX{} 在线服务:不想安装\TeX{} 发行版?}
  \begin{itemize}
    \item 无需安装,直接在浏览器中使用
    \item<+-> 方便多人协作预览、编写、批注
    \item<+-> 国际版\textbf{Overleaf} \link{https://www.overleaf.com}
      \begin{itemize}
        \item 预设模板丰富
        \item 免费版编译时长限制
      \end{itemize}
    \item<+-> 国内版\textbf{\TeX{} Page} \link{https://www.texpage.com}
      \begin{itemize}
        \item 预设众多中文字体
        \item 特色功能(暗黑模式等)
        \item 免费版编译时长限制
      \end{itemize}
    \item<+-> \textbf{协会自建Overleaf平台}(\url{latex.hlug.cn})
      \begin{itemize}
        \item 无编译时长限制
        \item 校内用户福利
      \end{itemize}
  \end{itemize}
\end{frame}
