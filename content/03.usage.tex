% !TeX root = ../main.tex

\section{\LaTeX{} 编写指南}

\begin{frame}[fragile]{Hello \LaTeX!}
\begin{texcode}[basicstyle=\small\ttfamily, xleftmargin=1.2cm,
    emph={[1]document}, emph={[2]article}]
  % 用 pdfLaTeX、XeLaTeX 或 LuaLaTeX 编译
  \documentclass{article}
  \begin{document}
  Hello \LaTeX!
  \end{document}
\end{texcode}
\pause
\begin{texcode}[basicstyle=\small\ttfamily, xleftmargin=1.2cm,
    emph={[1]document}, emph={[2]ctexart}]
  % 用 XeLaTeX 或 LuaLaTeX 编译
  \documentclass{ctexart}
  \begin{document}
  你好,\LaTeX!
  \end{document}
  \end{texcode}
\end{frame}

\begin{frame}{编译文档}
\begin{itemize}
  \item 现代 \TeX{} 引擎均可直接生成 PDF \pause
  \item 命令行
    \begin{itemize}
      \item |pdflatex|/|xelatex|/|lualatex| + |<文件名>[.tex]|
      \item 多次编译:读取并排版中间文件 \pause
      \item 推荐 \pkg{latexmk}:|latexmk [<选项>] [<文件名>]|
    \end{itemize} \pause

  \item 编辑器
    \begin{itemize}
      \item 按钮的背后仍然是命令
      \item |PATH| 环境变量:确定可执行文件的位置
      \item VS Code:配置\LaTeX{} Workshop插件的 |tools| \link{https://github.com/James-Yu/latex-workshop/wiki/Compile\#latex-tools} 和 |recipes| \link{https://github.com/James-Yu/latex-workshop/wiki/Compile\#latex-recipes} 设置
    \end{itemize}
\end{itemize}
\end{frame}


\begin{frame}[fragile]{语法}
\begin{itemize}
  \item 注释以 |%| 开头,忽略本行其后所有内容
  \item 命令以 |\| 开头的\emph{字母},区分大小写
    \begin{itemize}
      \item |\foo{arg}|:必选参数放在 |{...}| 中
      \item |\foo[bar]{arg}|:可选参数放在 |[...]|
    \end{itemize}

  \item 环境
    \begin{texcode}[gobble=4, emph={[1]envname}]
      \begin{envname}
        ...
      \end{envname}
    \end{texcode}

  \item 特殊符号需要转义:|\%|、|\$|、|\&|、|\textbackslash| 等
  \item 连续多个空格 = 单个空格 = 单个换行符 \pause
  \item \TeX{}/\LaTeX{} 的语法可以修改
\end{itemize}
\end{frame}


\begin{frame}[fragile]{文件结构}
  \begin{texcode}[moretexcs={\graphicspath}]
    \documentclass[a4paper]{ctexart}
    % 文档类型,如 ctexart,[]内是选项,如 a4paper
    % 这里开始是导言区
    \usepackage{graphicx} % 引用宏包
    \graphicspath{{figures/}} % 设置图片目录
    % 导言区到此为止
    \begin{document}
    这里开始是正文
    \end{document}
  \end{texcode}
\end{frame}


\begin{frame}{\LaTeX{} 常用命令}
  \begin{exampleblock}{简单命令}
  \centering
  \footnotesize
  \begin{tabular}{cccc}
    \cs{chapter} & \cs{section} & \cs{subsection} & \cs{paragraph} \\
    章 & 节 & 小节 & 带题头段落 \\\hline
    \cs{centering} & \cs{emph} & \cs{verb} & \cs{url} \\
    居中对齐 & 强调 & 原样输出 & 超链接 \\\hline
    \cs{footnote} & \cs{item} & \cs{caption} & \cs{includegraphics} \\
    脚注 & 列表条目 & 标题 & 插入图片 \\\hline
    \cs{label} & \cs{cite} & \cs{ref} & \cs{input},\cs{include}\\
    标号 & 引用参考文献 & 引用图表公式等 & 插入子文件\\\hline
    \end{tabular}
  \end{exampleblock}
  \begin{exampleblock}{环境}
  \centering
  \footnotesize
  \begin{tabular}{ccc}
    \env{table} & \env{figure} & \env{equation}\\
    表格 & 图片 & 公式 \\\hline
    \env{itemize} & \env{enumerate} & \env{description}\\
    无编号列表 & 编号列表 & 描述 \\\hline
  \end{tabular}
  \end{exampleblock}
\end{frame}


\begin{frame}[fragile]{文本标记}
\begin{itemize}
  \item 加粗:|{\bfseries ...}| 或 |\textbf{...}|
  \item 倾斜:|{\itshape ...}| 或 |\textit{...}|
  \item 字号:|\tiny|、|\small|、|\large|、|\Large| 等
  \item 换行:|\\|
  \item 缩进:|\indent|
  \item 居中:|\centering| 或 |center| 环境
\end{itemize}
\end{frame}

% \begin{frame}[fragile]{文本标记(二)}
% \begin{itemize}
%   \item 为什么要有不同的标记?\pause\mbox{}——表达不同的\alert{语义} \pause
%   \item |\textbf| 这样的命令是否表达语义? \pause
%   \item 再提一遍基本原则:\alert{内容与格式分离} \pause
%   \item 正确(或曰:合理)的做法

%     \begin{itemize}
%       \item 强调文字(意大利体):|\emph{...}|
%       \item 摘要(居中,小字号,带有标题):|abstract| 环境
%       \item 引用(左右边距较大):|quote| 或 |quotation| 环境
%       \item 自定义新的命令、环境
%     \end{itemize}\pause
%   \item 之前是否有在Word中新建样式呢?
% \end{itemize}
% \end{frame}


\begin{frame}[fragile]{常用环境:列表与枚举}
\begin{columns}
\begin{column}{0.54\textwidth}
  \begin{texcode}[gobble=4, emph={[1]enumerate,itemize}]
    \begin{enumerate}
      \item Frontend
        \begin{itemize}
          \item React
          \item Vue.js
          \item Svelte
        \end{itemize}
      \item Backend
        \begin{description}
          \item[PHP] Laravel
          \item[JavaScript] Express
          \item[Python] Django
        \end{description}
    \end{enumerate}
  \end{texcode}
\end{column}
\pause
\begin{column}{0.38\textwidth}
  \begin{enumerate}
    \item Frontend
      \begin{itemize}
        \item React
        \item Vue.js
        \item Svelte
      \end{itemize}
    \item Backend \par
        \vspace{0.5ex}\small
        \textbf{PHP}\enspace Laravel \\
        \textbf{JavaScript}\enspace Express \\
        \textbf{Python}\enspace Django
  \end{enumerate}
\end{column}
\end{columns}
\end{frame}


\begin{frame}[fragile]{常用环境:图片}
\begin{columns}
\begin{column}{0.64\textwidth}
  \begin{texcode}[gobble=4, moretexcs={\graphicspath,\includegraphics},
      emph={[1]figure}, emph={[2]graphicx}]
    % 不是 graphics
    \usepackage{graphicx}
    % 可以统一指定图片路径
    \graphicspath{{./figures/}}

    \begin{figure}
      \centering
      % 可指定宽度、高度等选项
      % 图片后缀名可省略,但建议保留
      \includegraphics[...]{hust-logo.pdf}
      \caption{华科校徽}
      \label{fig:hust-logo}
    \end{figure}
  \end{texcode}
\end{column}
\pause
\begin{column}{0.28\textwidth}
  \begin{figure}
    \centering
    \hustvi[width=\linewidth]{emblem}
    \caption{华科校徽}
    \label{fig:hust-logo_}
  \end{figure}
\end{column}
\end{columns}
\end{frame}


\begin{frame}[fragile]{常用环境:表格}
\begin{columns}
\begin{column}{0.62\textwidth}
  \begin{texcode}[gobble=4, moretexcs={\toprule,\midrule,\bottomrule},
      emph={[1]table,tabular}, emph={[2]booktabs}]
    \usepackage{booktabs} % 三线表
    \begin{table}
      \caption{中国人口调查}
      \label{tab:china-population}
      % 列格式:c 居中,l 左对齐,r 右对齐
      \begin{tabular}{cc}
        \toprule
          年份 & 人口 \\
        \midrule
          1953 &  6.0 \\
          ...
          2020 & 14.1 \\
        \bottomrule
      \end{tabular}
    \end{table}
  \end{texcode}
\end{column}
\pause
\begin{column}{0.3\textwidth}
  \begin{table}
    \caption{中国人口调查}
    \label{tab:china-population_}
    \footnotesize
    \begin{tabular}{cc}
      \toprule
        年份 & 人口 \\
      \midrule
        1953 &  6.0 \\
        1964 &  6.9 \\
        1982 & 10.1 \\
        1990 & 11.3 \\
        2000 & 12.7 \\
        2010 & 13.4 \\
        2020 & 14.1 \\
      \bottomrule
    \end{tabular}
  \end{table}
\end{column}
\end{columns}
\end{frame}

\begin{frame}[fragile]{浮动体与交叉引用}
\begin{itemize}
  \item<+-> 浮动体机制
    \begin{itemize}
      \item |figure| 和 |table| 环境,标题使用 |\caption| 命令
      \item 位置控制:|\begin{figure}[htb]|
      \item 希望浮动体不要乱跑:``这是病,得治''
            \link{https://liam.page/2017/03/11/floats-in-LaTeX-basic}
      \item 文本为主,图、表为辅
      \item 避免``见上图''、``见下表''
      \item 建议写完全文之后统一调整
    \end{itemize}

  \item<+-> 以标签控制交叉引用
    \begin{itemize}
      \item 被引处:|\label|
      \item 引用处:|\ref|、|\eqref|、|\autoref| 等(如图~\textcolor{hustblue}{\ref{fig:fudan-logo_}}、
            表~\textcolor{hustblue}{\ref{tab:china-population_}})
      \item 用有意义的标签:|\label{eq:euler-lagrange-eq}|
      \item \pkg{hyperref}:添加超链接、电子书签等
      \item 需多次编译——推荐 \pkg{latexmk}
    \end{itemize}
\end{itemize}
\end{frame}

\begin{frame}[fragile]{如何插图?}
\begin{itemize}
  \item<+-> 外部插入
    \begin{itemize}
      \item Mathematica、MATLAB
      \item PowerPoint、Visio、Adobe Illustrator、Inkscape、Figma 等
      \item Julia \pkg{Makie.jl}, \pkg{Plots.jl}、R \pkg{ggplot2}、Python \pkg{Matplotlib}、Plotly 等
      \item draw.io \link{https://www.draw.io}、
            Mathcha \link{https://www.mathcha.io}、
            ProcessOn \link{https://www.processon.com} 等网站
    \end{itemize}

  \item<+-> \TeX{} 内联
    \begin{itemize}
      \item \pkg{pgf}/\TikZ
      \item \pkg{pgfplots}
    \end{itemize}

  \item<+-> 插图格式
    \begin{itemize}
      \item 矢量图:|.pdf|
      \item 位图:|.jpg| 或 |.png|
      \item \alert{不再推荐 \texttt{.eps}}
      \item 不(完全)支持 |.svg|、|.bmp|
    \end{itemize}

  \item<+-> 参考:\link{https://www.zhihu.com/question/21664179}
                  \link{https://tex.stackexchange.com/q/158668}
                  \link{https://tex.stackexchange.com/q/72930}
\end{itemize}
\end{frame}

\begin{frame}[fragile]{文献}
\begin{itemize}
  \item 建议自动生成\pause

  \item |.bib| 数据库(条目会包含 key,用于引用)
    \begin{itemize}
      \item Google Scholar 复制,Zotero、Jabref 等生成
      \item 注意特殊符号、公式等常常需要人工检查
    \end{itemize} \pause

  \item 传统方法:\BibTeX{} 后端
    \begin{itemize}
      \item 指定样式:|\bibliographystyle{<style>}|(导言区)
      \item 标记引用:|\cite{<key>}|
      \item 插入参考文献:|\bibliography{<bib 文件>}|
      \item 更多文献、引用样式:\pkg{natbib} 宏包
      \item 国家标准 GB/T 7714--2015
            \link{https://www.gb688.cn/bzgk/gb/newGbInfo?hcno=7FA63E9BBA56E60471AEDAEBDE44B14C}
            \link{https://github.com/Haixing-Hu/GBT7714-2005-BibTeX-Style/files/153951/GBT.7714-2015.pdf}:
            \alert{\pkg{gbt7714} 宏包}
    \end{itemize} \pause

  \item 现代方法:\biber 后端 + \pkg{biblatex} 宏包
    \begin{itemize}
      \item 国家标准:\pkg{biblatex-gb7714-2015} 宏包
    \end{itemize} \pause

  \item 需多次编译——再次推荐 |latexmk|
\end{itemize}
\end{frame}

\begin{frame}[standout]
  \LaTeX{} 重头戏: 数学输入
\end{frame}

\begin{frame}[fragile]{数学模式}
\begin{itemize}
  \item 一切数学公式都要在数学模式下输入
    \begin{itemize}
      \item 不受外界字体命令控制
      \item 数学模式中空格不起作用,尽管用;但不能有空行
      \item 建议始终调用 \pkg{amsmath} 宏包 \pause
      \item \alert{不建议用 MathType 生成 \LaTeX{} 公式}
      \item 但可以用 MathJax \link{https://www.mathjax.org} 或 Ka\TeX{} \link{https://katex.org} 练习
    \end{itemize} \pause

  \item 行内\zhparen{inline}公式
    \begin{itemize}
      \item 用一对美元符号:|$...$|
      \item 示例:理想气体状态方程可以写为 $PV=nRT$, 其中 $P$、$V$ 和 $T$
        分别是压强、体积和绝对温度
    \end{itemize} \pause

  \item 行间\zhparen{display}公式
    \begin{itemize}
      \item 无编号:|\[...\]| 或 |equation*| 环境
      \item 编号:|equation| 环境
      \item \alert{不要用 \texttt{\$\$...\$\$}} \link{https://tex.stackexchange.com/a/69854}
    \end{itemize}
\end{itemize}
\end{frame}

\begin{frame}[fragile]{数学表达式}
\begin{itemize}
  \item<+-> 上下标
    \begin{itemize}
      \item |^| 和 |_|:
        \sout{\ttfamily\color{inline}f\^{}ab} vs |f^{ab}|,
        \sout{\ttfamily\color{inline}e\^{}x\^{}2} vs |{e^x}^2| 或 |e^{x^2}|
      \item 张量:|R^a{}_b{}^{cd}| 或使用 \pkg{tensor} 宏包
      \item 配合积分、求和、极限使用:|\int|、|\sum|、|\lim|;
        \lstinline[style=styleinline]|\(no)limits|
    \end{itemize}

  \item<+-> 分式
    \begin{itemize}
      \item |\frac{<分子>}{<分母>}|
      \item 行内分式、小分式不好看:改用 |a/b|,或改用独显公式
      \item \alert{不推荐 \texttt{\textbackslash dfrac}}
    \end{itemize}

  \item<+-> 根式
    \begin{itemize}
      \item |\sqrt[<次数>]{<内容>}|
      \item 复杂情况建议改用分数指数:|{...}^{1/n}|
    \end{itemize}

  \item<+-> 矩阵与行列式
    \begin{itemize}
      \item |matrix|、|pmatrix|、|vmatrix| 等环境
      \item 语法类似表格:|&| 分列,|\\| 换行
      \item 复杂矩阵:\pkg{nicematrix} 宏包
    \end{itemize}
\end{itemize}
\end{frame}

\begin{frame}[fragile]{数学括号与定界符}
\begin{itemize}
  \item<+-> 基本括号
    \begin{itemize}
      \item |(...)|、|[...]|、|\{...\}|、
      \item 绝对值 $\vert\cdot\vert$、范数 $\Vert\cdot\Vert$:\lstinline[style=styleinline]+|...|+ 或 |\vert...\vert|、|\Vert...\Vert|
      \item Dirac 符号 $\langle\cdot\rangle$:|\langle...\rangle|、\lstinline[style=styleinline]+|...\rangle+
    \end{itemize}

  \item<+-> 自动调节
    \begin{itemize}
      \item \lstinline[style=styleinline]+\left(...\mid|...\right)+ 等
      \item 大型括号是拼出来的
    \end{itemize}

  \item<+-> 手动调节
    \begin{itemize}
      \item 只有 4 + 1 档:|\big|、|\Big|、|\bigg|、|\Bigg|
      \item 声明左中右:|\bigl|、|\bigm|、|\bigr| 等
    \end{itemize}
\end{itemize}
\end{frame}

\begin{frame}[fragile]{多行公式}
\begin{itemize}
  \item 以下均要求 \pkg{amsmath} 宏包
  \item 独立数学环境
    \begin{itemize}
      \item 多行居中 |gather|、多行手动对齐 |align|、跨行 |multiline|
      \item 手动对齐:关系符前加 |&|
      \item 编号控制:|\tag{...}|、|\notag|
    \end{itemize}

  \item 内联数学环境
    \begin{itemize}
      \item 条件 |cases|、多行对齐 |split|、|...ed|
    \end{itemize} \pause

  \item 更复杂的情况
    \begin{itemize}
      \item \pkg{mathtools}、\pkg{empheq} 等
      \item 自动换行:\pkg{breqn}
      \item \alert{避免使用 \texttt{eqnarray} 环境} \link{https://tug.org/pracjourn/2012-1/madsen/madsen.pdf}
    \end{itemize}
\end{itemize}
\end{frame}

\begin{frame}[fragile]{精细调整}
\begin{itemize}
  \item 空格与间距
    \begin{itemize}
      \item |\quad|、|\qquad|、|\,|、|\!|
      \item \pkg{physics} 宏包:|\qq{<text>}|、|\qcomma|、|\qif| 等
    \end{itemize}

  \item 行内公式断行
    \begin{itemize}
      \item 默认只允许在运算符之后断行
      \item 不建议在行内插入过于复杂的公式
    \end{itemize}

  \item 多行公式
    \begin{itemize}
      \item 允许分页:|\allowdisplaybreaks|
      \item 前后间距:|\abovedisplay(short)skip|、|\belowdisplay(short)skip|
    \end{itemize} \pause

  \item 经验之谈
    \begin{itemize}
      \item 避免过度封装:\lstinline[style=styleinline]|\newcommand{\be}{\begin{equation}}|
      \item 不要浪费(现代)编辑器的高亮、提纲、预览、片段补充功能
    \end{itemize}
\end{itemize}
\end{frame}

\begin{frame}{宏包推荐}
\scriptsize
\setbeamertemplate{itemize/enumerate subbody begin}{\scriptsize}
\setlength{\leftmarginii}{1em}
\setlength{\columnsep}{2pt}
\begin{multicols}{3}
  \begin{itemize}
    \item 必备
      \begin{itemize}
        \item \pkg{amsmath}
        \item \pkg{graphicx}
        \item \pkg{hyperref}
      \end{itemize}

    \item 样式
      \begin{itemize}
        \item \pkg{caption}
        \item \pkg{enumitem}
        \item \pkg{fancyhdr}
        \item \pkg{footmisc}
        \item \pkg{geometry}
        \item \pkg{ntheorem}
        \item \pkg{titlesec}
      \end{itemize}

    \item 数学
      \begin{itemize}
        \item \pkg{bm}
        \item \pkg{mathtools}
        \item \pkg{physics}/\pkg{physics2}
        \item \pkg{unicode-math}
      \end{itemize}

    \item 表格
      \begin{itemize}
        \item \pkg{array}
        \item \pkg{booktabs}
        \item \pkg{longtable}
        \item \pkg{tabularx}
        \item \pkg{tabularray}
      \end{itemize}

    \item 插图、绘图
      \begin{itemize}
        \item \pkg{float}
        \item \pkg{pdfpages}
        \item \pkg{standalone}
        \item \pkg{subfig}
        \item \pkg{pgf}/\pkg{tikz}
        \item \pkg{pgfplots}
      \end{itemize}

    \item 字体
      \begin{itemize}
        \item \pkg{newtx}/\pkg{newpx}
        \item \pkg{pifont}
        \item \pkg{fontspec}
      \end{itemize}

    \item 多语言
      \begin{itemize}
        \item \pkg{babel}
        \item \pkg{polyglossia}
        \item \pkg{ctex}
        \item \pkg{xeCJK}
        \item \pkg{luatexja}
      \end{itemize}

    \item 杂项功能
      \begin{itemize}
        \item \pkg{algorithm2e}
        \item \pkg{beamer}
        \item \pkg{biblatex}
        \item \pkg{fancyhdr}
        \item \pkg{listings}
        \item \pkg{mhchem}
        \item \pkg{microtype}
        \item \pkg{minted}
        \item \pkg{natbib}
        \item \pkg{siunitx}
        \item \pkg{xcolor}
      \end{itemize}
  \end{itemize}
\end{multicols}
\vspace*{-0.5cm}
ACM出版系统允许使用的宏包 \link{https://authors.acm.org/proceedings/production-information/accepted-latex-packages}
\end{frame}
