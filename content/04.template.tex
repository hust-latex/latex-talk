% !TeX root = ../main.tex

\section{论文模板使用教程}

\begin{frame}[fragile]{模板}
\begin{itemize}
  \item<+-> 是什么?
    \begin{itemize}
      \item 设计好的格式框架
      \item 专注于内容:\alert{不必追求与期刊排版完全一致}
      \item Word 中的样式:``学好 \LaTeX{} 可以更科学地使用 Word''
    \end{itemize}

  \item<+-> 有哪些?
    \begin{itemize}
      \item 期刊:\pkg{acmart}、\pkg{elsarticle}、\pkg{IEEEtran}……
      \item 学位论文:\pkg{thuthesis}、\pkg{ustcthesis}、\pkg{SJTUThesis}……
    \end{itemize}

  \item<+-> 怎么用?
    \begin{itemize}
      \item |\documentclass[...]{...}|,配置参数,照常编写
      \item 可能与 \LaTeX{} 通常用法不同:\alert{|texdoc| 看文档}
    \end{itemize}

  \item<+-> 去哪里找?
    \begin{itemize}
      \item CTAN \link{https://ctan.org} 或 GitHub \link[\faGithub]{https://github.com}
      \item 期刊会议官网投稿指南
      \item ``湿兄用 U 盘 or 微信传给你的模板几乎一定是过时的''
    \end{itemize}
\end{itemize}
\end{frame}

\begin{frame}{论文排版}
  \begin{itemize}
    \item 获取模板
      \begin{itemize}
        \item 随发行版自带、手动网络下载
        \item 模板文档类 \texttt{.cls} 文件
        \item 示例 \texttt{.tex} 文件
      \end{itemize}
    \item 编辑 \texttt{.tex} 文件:添加用户内容
    \item 编译:生成 PDF 文档
  \end{itemize}
\end{frame}

\begin{frame}{论文排版举例}
  \begin{exampleblock}{IEEE 期刊论文\zhparen{\pkg{IEEEtran} 宏包}}
    \begin{itemize}
      \item 获取模板
        \begin{itemize}
          \item 发行版自带:将 \TeXLive 安装目录的 |texmf-dist/doc/latex/IEEEtran/bare_jrnl.tex| 复制到某个文件夹并编辑
          \item 手动下载:IEEE 官网 \link{https://www.ieee.org/conferences/publishing/templates.html}
        \end{itemize}
      \item 编译:\XeLaTeX{}、\pdfLaTeX{} 编译均可
    \end{itemize}
  \end{exampleblock}
  \begin{exampleblock}{ACM 期刊 / 会议论文\zhparen{\pkg{acmart} 宏包}}
    \begin{itemize}
      \item 获取模板
        \begin{itemize}
          \item 发行版自带:按需选择 |texmf-dist/doc/latex/acmart/samples| 下的各个样例 |.tex| 文件
          \item 手动网络下载:ACM 官网 \link{https://www.acm.org/publications/proceedings-template}
        \end{itemize}
      \item 编译:按样例文件要求选择程序\zhparen{通常 \pdfLaTeX{}}
    \end{itemize}
  \end{exampleblock}
\end{frame}

\begin{frame}{hustthesis: 非官方华中科技大学学位论文 \LaTeX{} 模板}
  \begin{itemize}
  \item 最早\LaTeXe{}版由许铖于2013年6月发布
  \item 2024年起由黄宇希以\LaTeX3重构并发布\pause

  \item 安装
    \begin{itemize}
      \item CTAN 官方 \link{http://mirrors.ctan.org/macros/latex/contrib/hustthesis.zip}(发行版可使用自带更新机制)
      \item GitHub \link[\faGithub\thesisgithubpath]{\thesisgithublink}
      \item 协会自建 Overleaf 平台 \link{\thesishlugoverleaf} 左上角 Menu $\rightarrow$ Copy Project
    \end{itemize} \pause

  \item 使用
    \begin{itemize}
      \item 下载解压使用 |hustthesis-user.zip| \link[\faGithub]{\thesisgithublink/releases/latest/download/hustthesis-user.zip}
      \item 参考示例文件 |demo/main.tex| \link[\faGithub]{\thesisgithublink/blob/master/demo/main.tex}
      \item 参考模板文档 |hustthesis.pdf| \link[\faGithub]{\thesisgithublink/releases/latest/download/hustthesis.pdf}
    \end{itemize}
  \end{itemize}
\end{frame}

\begin{frame}[fragile]{学位论文模板选项}
  \begin{description}
  \item[degree] 指定学位类型(本科/硕士/博士)
    \begin{texcode}
  \documentclass[degree=bachelor]{hustthesis}
    \end{texcode}
  \item[anonymous] 是否开启匿名评审
    \begin{texcode}
  \documentclass[degree=master,anonymous=true]{hustthesis}
    \end{texcode}
  \item[fontset] 指定字体(推荐使用 |windows|,详见模板文档说明)
    \begin{texcode}
  \documentclass[degree=doctor,fontset=windows]{hustthesis}
    \end{texcode}
  \end{description}
\end{frame}

\begin{frame}[fragile]{学位论文封面选项}
  使用 |\hustsetup{info=...}| 命令指定论文各类选项:
  \begin{columns}
    \begin{column}{.6\textwidth}
      \begin{table}[h]
      \centering
      \tiny
      \begin{tabular}{lcc}
        \toprule
        命令作用 & 中文对应选项 & 英文对应选项 \\
        \midrule
        论文标题 & |title| & |title*| \\
        作者姓名 & |author| &|author*| \\
        作者专业/类型 & |degree| &|degree*| \\
        学科名称 & |major| & |major*| \\
        导师 & |supervisor| & |supervisor*|\\
        日期 & \multicolumn{2}{c}{|date|}\\
        中图分类号 & |cls| & N/A \\
        学号 & |student-id| & N/A \\
        答辩委员会成员 & |committee| & N/A \\
        \bottomrule
      \end{tabular}
      \end{table}
    \end{column}
    \begin{column}{.35\textwidth}
      \begin{texcode}[gobble=8,keywords={hustsetup}]
        \hustsetup{
          info={
            title={标题},
            title*={Title},
            author={姓名},
            author*={Name},
            ...
          },
        }
      \end{texcode}
    \end{column}
  \end{columns}
\end{frame}
